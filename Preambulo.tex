%Archivo del preámbulo
\usepackage[utf8]{inputenc} %Ajustando codificacion de los caracteres 
\usepackage[spanish]{babel} %Para configurar le lenguaje
\usepackage{amsmath}
\usepackage{amsfonts}
\usepackage{amssymb}
\usepackage{makeidx} %Para crear índice alfabético
\usepackage{titlesec}
\titleformat{\chapter}[display]
{\normalfont\huge\bfseries}{}{0pt}{\Huge}
\makeindex
\usepackage{lipsum} %Para generar parrafos de ejemplo.
\usepackage{enumitem} %Para ajustar el estilo de los items enumerados.
\setlist[enumerate,1]{label=\arabic*.}
\usepackage{ragged2e} %Para justificar
\usepackage[top=2cm,bottom=2cm,inner=3cm,outer=2cm]{geometry} %Para modificar los margenes del documento.
%Seccion para teoremas y definiciones
\usepackage{amsthm}
\theoremstyle{plain}%definition,remark
\newtheorem{theorem}{Teorema}
\newtheorem{Def}{Definición}[section]
\newtheorem{Prop}[Def]{Proposición}
%-------------------------------------
%\usepackage{enumitem} %Para hacer ajustes a las listas
\usepackage{graphicx} %Paquete para insertar imagenes
\graphicspath{{./Imagenes/}}
\usepackage{float}% para ajustar la posición de un grafico o una tabla
%-------------------------------------
%Para referencias internas y externas
\usepackage{hyperref}
\hypersetup{
    colorlinks=true,
    linkcolor=blue,
    }
%Seccion para tablas
\usepackage{array}
\usepackage{multirow}
\usepackage{booktabs}%Para lineas especiales
\usepackage[table]{xcolor}%Para utilizar color
\usepackage{colortbl}%Para colo en tablas del entorno tabular
\usepackage{setspace}%Controlar espacio entre filas
\definecolor{color1}{RGB}{239,184,16}%Definiendo un color en el formato RGB
%-------------------------------------------
%Construyendo un título más elavorado.
%\maketitle
%Alineados, saltos de línea, entornos de centrado. 
%justify,raggedrigth,raggedleft,cenetering,begin{flushleft,flushright,center}
%ragged puede ser traducido como imperfecto. flush puede ser traducido como alineado.
%Comandos hfill, vfill, hspace, vspace.
%Tamaños de letra: tiny,scriptsize,footnotesize,small,normalsize,large,Large,huge,Huge.
%Listas, viñetas: enumerate, itemize, setlist[]{label=\arabic,\alph,\Alph,\roman,\Roman*.,}
%\setlist[enumerate,1]{label=\textbf{\arabic*.}} % Añade negrita a los números en el primer nivel
%Tipografias \textsf{•},\textrm{},\textit{},\texttt{•},\textsc{}mayusculas,\textbf{},\textmd{},\bfseries,\mdserie s,\textsl{Roman inclinado} pequeñas, Mecanografiado,ttfamily, \rmfamily, \sffamily, \itshape,\upshape,\slshape
%\mainmatter
%\pagenumbering{arabic}
%area de encabezados del libro
\usepackage{fancyhdr}
\usepackage{afterpage}
\pagestyle{fancy}
\fancyhf{}%Limpia los encabezados predeterminados
\fancyhead[RO,RE]{\thepage}%LE: left even (izquierda, par), RO: right odd (derecha, impar)
\fancyhead[LO]{\nouppercase{\leftmark}}% Titulo del 	capítulo
\fancyhead[LE]{\nouppercase{\rightmark}}% Titulo de las secciones
\fancyhead[C]{Editorial}
\fancyfoot[LE,RO]{Jose Alvarenga}
\fancyfoot[LE]{Editorial}
\renewcommand{\footrule}{{\textcolor{black}{\rule{\headwidth}{0.1cm}}}}
\renewcommand{\headrule}{{\vskip-0.3cm\textcolor{black}{\rule{\headwidth}{0.1cm}}}}