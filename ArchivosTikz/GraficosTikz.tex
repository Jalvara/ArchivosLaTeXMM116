\documentclass[twoside,11pt,letterpaper]{article}	
\usepackage[utf8]{inputenc} %Ajustando codificacion de los caracteres 
\usepackage[spanish]{babel} %Para configurar le lenguaje
\usepackage{amsmath}
\usepackage{amsfonts}
\usepackage{amssymb} 
\usepackage{tikz}
\usepackage{makeidx} %Para crear índice alfabético
\usepackage{lipsum} %Para generar parrafos de ejemplo.
\usepackage{enumitem} %Para ajustar el estilo de los items enumerados.
\setlist[enumerate,1]{label=\arabic*.}
\usepackage{ragged2e} %Para justificar
\usepackage[margin=2.5cm]{geometry} %Para modificar los margenes del documento.
%Seccion para teoremas y definiciones
\usepackage{amsthm}
\theoremstyle{plain}%definition,remark
\newtheorem{theorem}{Teorema}
\newtheorem{Def}{Definición}[section]
\newtheorem{Prop}[Def]{Proposición}
%-------------------------------------
%\usepackage{enumitem} %Para hacer ajustes a las listas
\usepackage{graphicx} %Paquete para insertar imagenes
\graphicspath{{./Imagenes/}}
\usepackage{float}% para ajustar la posición de un grafico o una tabla
%-------------------------------------
\usepackage{hyperref}
\title{\Huge Universidad Nacional Autónoma de Honduras\\
\huge Figuras construidas en TikZ}
\author{Jose Alvarenga}
\makeindex
\begin{document}
%Construyendo un título más elavorado.
\maketitle
\begin{figure}[H]
\centering
\caption{En este apartado dibujaremos una línea}
\begin{tikzpicture}
\draw[fill=blue!50,draw=red] (0,0) circle (1cm);
\filldraw (0,0) circle (1pt);
\draw (0,0)--(0,1);
\draw (1,0)--(2,0)--(2,1)--(1,1)--(1,0);
\end{tikzpicture}
\end{figure}
\begin{figure}[H]
\centering
\caption{En este apartado dibujaremos una línea curva}
\begin{tikzpicture}[scale=2,rotate=15]
\draw[step=0.5cm,draw=gray!50] (0,0) grid (3,2);
\draw (0,0)..controls(1,1.5) and (2,2)..(3,2);
\draw (0,0)--(0,2);
\draw (0,0)--(2,0);
\draw (0,0) rectangle (3,2);
\end{tikzpicture}
\end{figure}
\begin{figure}[H]
\centering
\caption{En este apartado dibujaremos una línea curva}
\begin{tikzpicture}[scale=2,rotate=0]
\draw (3,2) arc (25:100:2cm);
\end{tikzpicture}
\end{figure}
\end{document}
